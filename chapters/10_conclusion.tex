% !TeX root = ../main.tex
% Add the above to each chapter to make compiling the PDF easier in some editors.

\chapter{Conclusion}\label{chapter:conclusion}
This thesis studied the relationship between stress and eating behaviors, and developed methods to predict individuals' eating patterns under the influence of stress, using a chatbot called Demezys. In order to collect user data, the chatbot was built with features including detecting stress via adaptive sampling, responding and processing users' self-reporting of stress, and guiding users to record eating behavior data. It was developed using the Rasa platform, which is an open-source platform for developing conversational agents based on natural language understanding, and tested with the Telegram Bot API.

A conversation flow was designed such that users would be prompted several times per day to record his or her stress level and food eaten whilst being stressed. In the experiment, the participants were divided into two groups, one of which received the adaptive scheduling of messages, where stress detection happened at the time of the day when they were likely to be more stressed based on historical data. Stress levels, together with the amount of food eaten compared with the non-stressed state, the number of pieces consumed per day, and food descriptions were collected and recorded.

At the end of the data collection, four candidates were selected for further user studies. They were invited to label their food description items as either comfort food or not. Afterward, their data was preprocessed and multiple models were trained using traditional machine learning methods for multiclass classification, from which the best performing models were selected for evaluation. Eating behaviors including the amount of food and comfort food eaten under stress were predicted based on several input conditions, which were transformed into stress-eating reports and evaluated by the candidates.

Based on both the accuracy of the models and candidate evaluations, the methods were proven to be able to predict users' eating behaviors given stress levels. There were, however, certain limitations of this work, including the limited amount of data used for training and the decreased user attachment over time. Future work was suggested in both extensive user studies for building better models and also developing virtual dietary advisors and food recommendation systems based on the result of this work.
