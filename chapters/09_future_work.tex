% !TeX root = ../main.tex
% Add the above to each chapter to make compiling the PDF easier in some editors.

\chapter{Future Work}\label{chapter:future_work}
With both the outcome and limitations presented in previous chapters, this chapter gives suggestions on future work that can be based on this thesis.\bigskip

\noindent As mentioned in the previous chapter, the limited amount of data was the biggest issue in this project. Therefore, long-term user studies could be conducted to track the stress and eating patterns of participants in a longer period of time. The data collected from the long-term study could be used to conduct feature selections, model training and predictions just as was done in this work, to further prove the robustness of the models and the feasibility of the methods demonstrated for building stress-eating profiles.\bigskip

\noindent Since user attachment has been another issue of the chatbot developed in this project, more time and effort could be investigated in building a more personalized chatbot that can interact with its users in a more interesting way. An example of this would be a dietary adviser that gives its user advice based on his or her stress level. As discussed in \autoref{chapter:goal_req}, the goal of this project was to build a foundation for potential dietary advisers. Now that the foundation is laid, it is advised that systems could be developed based on that. In addition, researches could be done in the area of food substitution. A typical scenario would be that after grasping the user's stress-eating pattern, the virtual adviser understands that the user tends to eat more under stress, predicts the type of food the user is likely to eat, and in case it's comfort food with high energy, suggests a substitution of it which resembles its appearance but contains lower energy.\bigskip

\noindent It is also worth noticing that the stress-eating statements in the surveys used for user evalutation were manually written based on the prediction results, which is not scalable to a larger amount of users. Future research is needed on interpreting the results of the prediction and automatically generating statements and stress-eating reports using techniques in the fields such as natural language generation (NLG).\bigskip

\noindent Last but not the least, since Rasa provides custom actions which offer high degrees of freedom including calling other servers, it is imaginable to develop an API that retrieves stress information collected via analyzing smart device (e.g. smart watch) inputs or social media context, which can be connected with the chatbot platform developed in this research. In that manner, the result of any other research into stress detection, which potentially offers more accuracy, improved richness of data, and more possibilities for feature engineering, could be integrated into this platform, for further studies into the relationship between stress and eating as well as developing better food recommendation systems.
