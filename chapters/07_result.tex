% !TeX root = ../main.tex
% Add the above to each chapter to make compiling the PDF easier in some editors.

\chapter{Result}\label{chapter:result}
Since the result of data collection via the chatbot was already demonstrated in the previous chapter, this chapter focuses on the result of cross-validation, user evaluation, and the effectiveness of adaptive sampling.

The results of the model selection for each task and its predictions for each user per task are showin in \autoref{model-selection}. As can be seen, a majority of the selected models achieved accuracies close to or over 75 percent. However, despite the relatively high cross-validation score, it is still likely that the models were overfitted, given the fact that the data size is small, and consequently neither the training or testing set in each phase can be representative enough to the user (\cite{45_overfitting}). Therefore, the user evaluation provided important additional input on how well the models as well as the methods presented in this work performed. For the four candidates, 16 statements were evaluated, and the evaluation score was 15. The average score per statement was, therefore, close to 1, corresponding to the evaluation "Yeah, sometimes". 7 out of the 16 statements were given the response "Oh my word! How did you know that!", demonstrating that, in general, the candidates thought that the stress-eating profiles captured their stress-eating patterns correctly, despite the relatively short period of data collection and small sample size. This shows that the methods developed in this thesis are potentially feasible for creating eating behavior reports for a larger population with more data.\bigskip

\begin{lstlisting}[label={model-selection},caption={Result of model selection and prediction for each task},captionpos=b]
**** Result for Daily more\_or\_less ****

SVM accuracy for candidate 2:	0.7533333333333333
High-stress prediction for candidate 2:	  ['less']
Acute-stress prediction for candidate 2:	['less']
Low-stress prediction for candidate 2:	  ['less']

DT accuracy for candidate 1:	0.75
High-stress prediction for candidate 1:	  ['more']
Acute-stress prediction for candidate 1:	['more']
Low-stress prediction for candidate 1:	  ['more']

kNN accuracy for candidate 3:	0.96
High-stress prediction for candidate 3:	  ['same']
Acute-stress prediction for candidate 3:	['same']
Low-stress prediction for candidate 3:	  ['same']

**** Result for Daily pieces ****

DT accuracy for candidate 2:	0.7328571428571429
High-stress prediction for candidate 2:	  [3]
Acute-stress prediction for candidate 2:	[3]
Low-stress prediction for candidate 2:	  [4]
Unstressed prediction for candidate 2:	  [3]

kNN accuracy for candidate 1:	0.5742857142857143
High-stress prediction for candidate 1:	  [7]
Acute-stress prediction for candidate 1:	[7]
Low-stress prediction for candidate 1:	  [6]
Unstressed prediction for candidate 1:	  [6]

SVM accuracy for candidate 3:	0.8016666666666667
High-stress prediction for candidate 3:	  [3]
Acute-stress prediction for candidate 3:	[3]
Low-stress prediction for candidate 3:	  [3]
Unstressed prediction for candidate 3:	  [2]

kNN accuracy for candidate 4:	0.7571428571428571
High-stress prediction for candidate 4:	  [3]
Acute-stress prediction for candidate 4:	[3]
Low-stress prediction for candidate 4:	  [1]
Unstressed prediction for candidate 4:	  [3]

**** Result for Daily comfort\_food ****

kNN accuracy for candidate 2:	0.6833333333333332
High-stress prediction for candidate 2:	  ['False']
Acute-stress prediction for candidate 2:	['False']
Low-stress prediction for candidate 2:	  ['False']

SVM accuracy for candidate 1:	0.6866666666666668
High-stress prediction for candidate 1:	  ['True']
Acute-stress prediction for candidate 1:	['True']
Low-stress prediction for candidate 1:	  ['True']

DT accuracy for candidate 3:	0.96
High-stress prediction for candidate 3:	  ['True']
Acute-stress prediction for candidate 3:	['True']
Low-stress prediction for candidate 3:	  ['True']

**** Result for Event-Based comfort\_food ****

DT accuracy for candidate 1:	0.8099999999999999
High-stress prediction for candidate 1:	  ['True']
Acute-stress prediction for candidate 1:	['True']
Low-stress prediction for candidate 1:	  ['True']
\end{lstlisting}

\bigskip
In addition to the methods of building the stress-eating profile, the adaptive sampling scheme was also evaluated. This was done by counting the number of \emph{tell\_stress\_level} entries (stress entries). In the first half (2 weeks) of the data collection, 195 stress entries were recorded for all participants, with 143 of which from participants with even \emph{chat\_id}s, consisting of 73 percent of the total entries. In the second half, 124 entries were recorded, among which 104 were from those with even \emph{chat\_id}s. The proportion was increased to 84 percent. In addition, despite the decrease in the total number of entries, the number of entries of two participants with even \emph{chat\_id}s increased in the second half, which is not to be seen by any user with an odd ID. This suggests that the adaptive sampling scheme may have been effective, resulting in more accurate detection of stress. However, since both the population and the duration of this study were limited, further studies are needed to prove this hypothesis.

Meanwhile, the decrease in the total number of entries was expected, as users tend to be less attached to the chatbot for two reasons. One is that some users might have tried the chatbot out of curiosity (\cite{47_curiosity}), and such curiosity decreases over time. The other is, as \citeauthor{48_why_chatbot} (\citeyear{48_why_chatbot}) suggested, that the biggest motivation people interact with chatbots is for productivity, which in the case of Demezys, is hard to obtain (users hardly get any input and immediate help by telling the chatbot whether he or she is stressed).
