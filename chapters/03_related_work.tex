% !TeX root = ../main.tex
% Add the above to each chapter to make compiling the PDF easier in some editors.

\chapter{Related Work}\label{chapter:related_work}
This chapter presents some of the related work focusing on topics concerning this project, as discussed in the previous chapters. This includes work in the areas of the effect of emotions and stress on eating, chatbots as dietary advisors and other forms of behavioral change, stress detection using smartphones, stress detection using wearable smart devices, and the intrusiveness and timing of digital intervention, followed by a brief review of a similar study which uses dialogue systems to assess stress eating.

\section{The Effect of Emotions and Stress on Eating}
As was discussed in \autoref{chapter:introduction}, both emotions and stress affect eating choices. This topic has been widely researched in a multitude of fields, including psychology, nutrition, and computer science.

\subsection{The Effect of Emotions on Eating}
\citeauthor{4_mood_eat} (\citeyear{4_mood_eat}) investigated how mood influences food choice. Here, mood is treated as an equivalence of emotion. The paper paid particular attention to the different mechanisms with which positive and negative moods affect food choices, pointing out that positive moods are more likely to benefit long-term goals such as health, resulting in choices of healthy food, while negative moods cue the need of relief from such moods, leading to choices of indulgent foods which often contain high levels of calories. The latter is physiologically plausible since foods with high levels of fat and sugar trigger the release of endorphins which helps with creating affectionate and happy mood (\cite{32_endorphins}).\bigskip

\noindent \citeauthor{16_martin} (\citeyear{16_martin}) also researched on the relations between an individual's emotional state and eating behavior, focusing on which specific emotions impact eating behavior the most. He made a clearer distinction among affects, emotions, and moods, which are otherwise often used interchangeably in research works. "Core affect" is defined as a rather general feeling, e.g. a sense of pleasure or displeasure. Emotions are caused by external or internal events and are triggered immediately after the events. In contrast, moods often last longer and do not necessarily connect to specific events. Based on these definitions, \citeauthor{16_martin} conducted a survey among nutrition and diet experts, finding that emotions and moods which are on the positive and neutral spectrum of core affects are less likely to relate with eating behaviors. These moods and emotions include calm/relaxed, joy, hostility, self-assurance, confusion, and excitement. On the contrary, the most affecting emotions and moods on eating turned out to be frustrated/irritated, anger, tense, fear, sad, hopeless, bored, tired, and guilt.

\subsection{The Effect of Stress on Eating}
\citeauthor{5_stress_eating} (\citeyear{5_stress_eating}) explored the widely accepted belief that stress influences human eating behavior, researching on the mechanism of such influence. They found that stress indeed influences eating patterns in humans. In case of such influences, stress can alter one's food intake in two ways, resulting in either over-eating or under-eating, which could be influenced by the severity of the stressor. Additionally, like \citeauthor{4_mood_eat}, \citeauthor{5_stress_eating} looked into not only the immediate effect of such influence, but also the long-term effect, concluding that chronic life stress might lead to a stronger preference towards energy and nutrient-dense foods, which in turn is one of the causes of weight gain, and in worse cases, obesity. Demographic factors were also taken into consideration, where it appeared that such effect is more significant in men than women.
