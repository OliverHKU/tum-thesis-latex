% !TeX root = ../main.tex
% Add the above to each chapter to make compiling the PDF easier in some editors.

\chapter{Introduction}\label{chapter:introduction}
Eating is an activity that people perform on a daily basis. It is the essential source of ingredients for us humans. Our nutrition intake, in turn, affects our health. However, people’s choice of food cannot be simply regulated in terms of time and ingredients to make the best health effect out of it, because it is a highly emotional behavior (\cite{4_mood_eat}). According to \citeauthor{4_mood_eat}, both positive and negative moods affect food choices. Especially, having negative moods often leads one to pick indulgent food instead of healthy food to cope with the emotion.

Stress is a common reaction to the environment that is often linked to negative emotions. In fact, \citeauthor{1_stress_emotion} (\citeyear{1_stress_emotion}) suggests that there is a significant positive correlation between the level of stress one has and the degree of negative emotions one experiences. Combining the results from both studies, it is therefore highly likely that food choices could be affected by stress.

A study by \citeauthor{2_many_stress} (\citeyear{2_many_stress}) suggests that a majority of the population in the United Kingdom may have been overwhelmed with stress at some time within the year 2018. This suggests that many of the health problems resulted from unhealthy eating behaviors could be linked to stress. However, regulating eating behavior often requires a deep understanding of nutrition and diet, which is not the possession of non-experts. There are professionals who are out there to offer counseling services on people’s diet, but this is understandably not always accessible by the general public, given the pervasiveness of stress among them. Moreover, the specific eating behavior resulted from stress differs among individuals (\cite{5_stress_eating}). For example, the same level of stress can lead to overeating for one person, but undereating for another. It is, therefore, crucial to work out an individual’s eating behavior under the influence of stress without professional medical intervention. This information can be helpful in building food recommendation systems that can detect stress, and recommend healthy food based on the user’s eating patterns. The prerequisite of such is to build another (predictive) system so that given a specific user and his/her stress level, it can predict what the user is likely to eat, especially whether he/she is likely to eat more or less than usual. This thesis focuses on establishing a method to build such a system. \bigskip

\noindent The first step is to collect user data. Specifically, data on users’ stress and eating behavior, as well as the relations between them. One way of doing this is to use a chatbot. Compared with more explicit ways of acquiring data, such as questionnaires or interviews, a chatbot is obviously less intrusive and offers the possibility to collect data in a real-world setting instead of in laboratories. On the other hand, users are likely to be more adherent to chatbots compared to other types of cognitive-behavioral therapeutic (CBT) apps such as self-help web-based therapy (\cite{6_cbt}) given their conversational and human-like nature, which is crucial in the context of this research (\cite{7_woebot}).

In this thesis, we present the design, implementation, and testing of a chatbot which collects data on the users’ stress information and food consumed whilst being stressed, and presents a method to build a connection between the two on a per-user basis, i.e. building a stress-eating profile for a potential user.

\section{Section}

\subsection{Subsection}

See~\autoref{tab:sample}, \autoref{fig:sample-drawing}, \autoref{fig:sample-plot}, \autoref{fig:sample-listing}.

\begin{table}[htpb]
  \caption[Example table]{An example for a simple table.}\label{tab:sample}
  \centering
  \begin{tabular}{l l l l}
    \toprule
      A & B & C & D \\
    \midrule
      1 & 2 & 1 & 2 \\
      2 & 3 & 2 & 3 \\
    \bottomrule
  \end{tabular}
\end{table}

\begin{figure}[htpb]
  \centering
  % This should probably go into a file in figures/
  \begin{tikzpicture}[node distance=3cm]
    \node (R0) {$R_1$};
    \node (R1) [right of=R0] {$R_2$};
    \node (R2) [below of=R1] {$R_4$};
    \node (R3) [below of=R0] {$R_3$};
    \node (R4) [right of=R1] {$R_5$};

    \path[every node]
      (R0) edge (R1)
      (R0) edge (R3)
      (R3) edge (R2)
      (R2) edge (R1)
      (R1) edge (R4);
  \end{tikzpicture}
  \caption[Example drawing]{An example for a simple drawing.}\label{fig:sample-drawing}
\end{figure}

\begin{figure}[htpb]
  \centering

  \pgfplotstableset{col sep=&, row sep=\\}
  % This should probably go into a file in data/
  \pgfplotstableread{
    a & b    \\
    1 & 1000 \\
    2 & 1500 \\
    3 & 1600 \\
  }\exampleA
  \pgfplotstableread{
    a & b    \\
    1 & 1200 \\
    2 & 800 \\
    3 & 1400 \\
  }\exampleB
  % This should probably go into a file in figures/
  \begin{tikzpicture}
    \begin{axis}[
        ymin=0,
        legend style={legend pos=south east},
        grid,
        thick,
        ylabel=Y,
        xlabel=X
      ]
      \addplot table[x=a, y=b]{\exampleA};
      \addlegendentry{Example A};
      \addplot table[x=a, y=b]{\exampleB};
      \addlegendentry{Example B};
    \end{axis}
  \end{tikzpicture}
  \caption[Example plot]{An example for a simple plot.}\label{fig:sample-plot}
\end{figure}

\begin{figure}[htpb]
  \centering
  \begin{tabular}{c}
  \begin{lstlisting}[language=SQL]
    SELECT * FROM tbl WHERE tbl.str = "str"
  \end{lstlisting}
  \end{tabular}
  \caption[Example listing]{An example for a source code listing.}\label{fig:sample-listing}
\end{figure}
