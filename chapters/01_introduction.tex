% !TeX root = ../main.tex
% Add the above to each chapter to make compiling the PDF easier in some editors.

\chapter{Introduction}\label{chapter:introduction}
Eating is an activity that people perform on a daily basis. It is the essential source of ingredients for us humans. Our nutrition intake, in turn, affects our health. However, people’s choice of food cannot be simply regulated in terms of time and ingredients to make the best health effect out of it, because it is a highly emotional behavior (\cite{4_mood_eat, 14_comfort_food}). According to \citeauthor{4_mood_eat}, both positive and negative moods affect food choices. Especially, having negative moods often leads one to pick indulgent food instead of healthy food to cope with the emotion.

Stress is a common reaction to the environment that is often linked to negative emotions. In fact, \citeauthor{1_stress_emotion} (\citeyear{1_stress_emotion}) suggests that there is a significant positive correlation between the level of stress one has and the degree of negative emotions one experiences. Combining the results from both studies, it is therefore highly likely that food choices could be affected by stress.

A study by \citeauthor{2_many_stress} (\citeyear{2_many_stress}) suggests that a majority of the population in the United Kingdom may have been overwhelmed with stress at some time within the year 2018. This suggests that many of the health problems resulted from unhealthy eating behaviors could be linked to stress. However, regulating eating behavior often requires a deep understanding of nutrition and diet, which is not the possession of non-experts. There are professionals who are out there to offer counseling services on people’s diet, but this is understandably not always accessible by the general public, given the pervasiveness of stress among them. Moreover, the specific eating behaviors resulted from stress differ among individuals (\cite{5_stress_eating}). For example, the same level of stress can lead to overeating for one person, but undereating for another. It is, therefore, crucial to work out an individual’s eating behavior under the influence of stress without professional medical intervention. This information can be helpful in building food recommendation systems that can detect stress, and recommend healthy food based on the user’s eating patterns. The prerequisite of such is to build another (predictive) system so that given a specific user and his/her stress level, it can predict what the user is likely to eat, especially whether he/she is likely to eat more or less than usual. This thesis focuses on establishing methods to build such a system. \bigskip

\noindent The first step is to collect user data. Specifically, data on users’ stress and eating behavior, as well as the relations between them. One way of doing this is to use a chatbot. Compared with more explicit ways of acquiring data, such as questionnaires or interviews, a chatbot is obviously less intrusive and offers the possibility to collect data in a real-world setting instead of in laboratories. On the other hand, users are likely to be more adherent to chatbots compared to other types of cognitive-behavioral therapeutic (CBT) apps such as self-help web-based therapy (\cite{6_cbt}) given their conversational and human-like nature, which is crucial in the context of this research (\cite{3_woebot}).

This thesis presents the design, implementation, and testing of a chatbot which collects data on the users’ stress information and food consumed whilst being stressed, and presents methods to build a connection between the two on a per-user basis, i.e. building a stress-eating profile for a potential user. The following chapters will provide details regarding the design, realization, and evaluation of such a system. \hyperref[chapter:goal_req]{Chapter~\ref*{chapter:goal_req}} formalizes the goal and requirements of the project. It will put the terms “stress detection”, “dialogue-based self-reporting of stress” and “eating behaviors” into the context of this work. \hyperref[chapter:related_work]{Chapter~\ref*{chapter:related_work}} summarizes previous works related to the topic of this research, which focused on the effect of emotion on eating, chatbots as dietary advisors, and stress detection utilizing data collected by smartphones and wearable smart devices. \hyperref[chapter:sys_design]{Chapter~\ref*{chapter:sys_design}} offers the details of the system design, focusing on the framework used for building the chatbot, the detection and self-reporting of stress, the collection and processing of eating behavior data, the design of states in the conversation flow, and how data is persisted for processing and analysis. \hyperref[chapter:experiment]{Chapter~\ref*{chapter:experiment}} presents how the chatbot was tested among pilot users, including the recruitment process, the process of data collection, and the methods for evaluation of such experiment. This will be followed by \autoref{chapter:data_analysis} which explains the data gathered, including the data size and content, as well as how the data is trained to result in predictive models. \hyperref[chapter:result]{Chapter~\ref*{chapter:result}} demonstrates the result of both the data processing and the feedback from the participants.

There are certain limitations of this research which are discussed in \autoref{chapter:limitations}. Since this study aims at building a foundation for more sophisticated systems related to stress eating, e.g. food recommendation systems, \autoref{chapter:future_work} is going to give suggestions to future work. Finally, \autoref{chapter:conclusion} will conclude this thesis.
