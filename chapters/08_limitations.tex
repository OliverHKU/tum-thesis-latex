% !TeX root = ../main.tex
% Add the above to each chapter to make compiling the PDF easier in some editors.

\chapter{Limitations}\label{chapter:limitations}
Despite achieving its goal in building a stress-eating profile for random users, this study has certain limitations, which are discussed in this chapter.

The most evident limitation, as have mentioned in previous chapters, is the size of the data. This resulted from both the limited time in the user study and the design of the chatbot. First, the user study was conducted for four weeks, meaning that there could be up to 28 day-level entries per user. This is not enough for building models that are representative enough. Second, as shown in \autoref{chapter:sys_design}, the conversation flow was rule-based, meaning that the chatbot is not robust enough to handle users' input which is out of the context. This, on one hand, allows the chatbot to focus on collecting data it needed accurately, but on the other, makes it less interesting to the users. This could be one of the reasons why user interactions with the chatbot decreased over time.

Another limitation was that users were reluctant to actively report stress, with almost all the \emph{tell\_stress\_level} entries resulted from initiatives taken by the chatbot. One of the reasons for this drawback is again the fact that the chatbot is not "interesting" enough. Additionally, as \citeauthor{48_why_chatbot} (\citeyear{48_why_chatbot}) suggested, people are concerned with their productivity when using a chatbot, and reporting that he or she is stressed clearly does not help with that. Based on the analysis, one potential solution is to generate a stress-eating report on the flight that can be retrieved by the user at any time, which consists of the visualization of the user's stress level and eating behavior history. This might motivate the user to communicate more actively to the chatbot.

Furthermore, although there's evidently a strong connection between stress and eating, there are other factors that can affect eating choices. These factors might include the time of the day, being on travel (not being able to access domestic food), or specific diet plans. For example, one of the participants gave a piece of feedback, saying that she is practicing a diet strictly required by her religion, where she always eats the same amount of food every day. Additionally, since this research coincided with the current situation of the global lockdown, all candidates reported self-cooking frequently, which may not be the case in a normal setting when people have more choices of eating out.
